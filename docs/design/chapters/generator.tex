\chapter{State Proof Generator}
\label{section:system}

The basic idea is to wrap the original proofs provided by DBMS replication 
protocol trustless I/O extension-based cluster to the StarkNet-compatible one. 

Using these proofs, StarkNet/StarkEx (and Cairo in general) users will become 
capable to rely on third-party clustered databases data in a trustless manner by 
receiving \texttt{=nil;} DBMS query data, and a STARK proof, proving its structure 
and contents. This would provide StarkNet/StarkEx users/developers with bridging
capabilities with any protocol \texttt{=nil;} DBMS supports (e.g. Bitcoin, Solana, others).

Remind that there are two types of proofs in \texttt{=nil;} DBMS Cluster:
\begin{itemize}
    \item State Proofs provide compressed state updates from the tracked clusters.
            The circuits of these proofs are fixed and known in advance.
    \item Query Proofs show that some data is contained in the compressed state.
            The query proofs' details are not known in advance.
            However, they can be represented by relatively simple circuits that can be 
            generated in runtime.
\end{itemize}

Some additional setup is required 
to integrade a tracked cluster into StarkNet via \texttt{=nil;} DBMS Cluster
\begin{enumerate}
    \item Prepare Placeholder verification circuit for state proofs in PLONK-arithmetization.
    \item Translate this circuit into AIR StarkNet-compatible representation. 
\end{enumerate}

The wrapping mechanism slightly change intercluster transcation \textbf{Read} operation\footnote{
    For details on intercluster transaction process, see 
    \url{https://dbms.nil.foundation/io.pdf}}.

\begin{algorithm}[h]
\caption{Specialized Read Operation}
\textbf{Input}: $\texttt{from}, \texttt{query}, \texttt{need\_proof}$
\begin{enumerate}
	\item Generate circuit $C$ from $\texttt{query}$.
	\item Query data $\texttt{response}$ required by $\texttt{query}$ from database $\texttt{from}$.
	\item If $\texttt{need\_proof} = 1$ (otherwise $\pi_{\texttt{state}} = \pi_{\texttt{resp}} = 0$):
	\begin{enumerate}
		\item If there is not anchor for the state, 
            generate Placeholder proof $\pi'_{\texttt{state}}$ of the consistent state $\texttt{state}$ of $\texttt{from}$
		\item Generate STARK proof $\pi_{\texttt{resp}}$ of $C$.
		\item Generate STARK proof $\pi_{\texttt{state}}$ that verifies $\pi'_{\texttt{state}}$.
	\end{enumerate}
	\item Write $(\pi_{\texttt{state}}, \pi_{\texttt{resp}}, \texttt{response})$ to $\mathcal{M}$.
			Validators of $\mathcal{M}$ can verify $\texttt{response}$ using either $\pi_{\texttt{state}}, \pi_{\texttt{resp}}$ or the tracked state of $\texttt{from}$. 
\end{enumerate}
\end{algorithm}

\section{Circuit Conversion}

We can move arithmetic constraints from PLONK into AIR representation straightforwardly. 
However, STARKs do not provide separate Permutation and Lookup arguments.
It means that we need to convert copy- and lookup-constraints into 
basic AIR constraints.  
\begin{enumerate}
    \item Split gates into separate constraints.
    \item Represent each constraint as AIR constraint on the rows with the rows where selector values are equal to 1.
    \item Represent copy-constraint as AIR constraints with the corresponding shift. 
    \item Represent lookup-constraints other separate lookup-argument as AIR constraints with the inner lookup table. 
\end{enumerate}
